\documentclass[12pt,a4paper]{scrartcl} 
\usepackage[utf8]{inputenc}
\usepackage[английский,русский]{babel}
\usepackage{indentfirst}
\usepackage{misccorr}
\usepackage{graphicx}

\usepackage{amsmath,amsfonts,amssymb,amsthm,mathtools} % Это AMS
\начать{документ}

	\begin{titlepage}
		\begin{центр}
			\большой
 МИНИСТЕРСТВО НАУКИ И ВЫСШЕГО ОБРАЗОВАНИЯ РОССИЙСКОЙ ФЕДЕРАЦИИ
			
 Федеральное государственное бюджетное образовательное учреждение высшего образования
			
			\textbf{АДЫГЕЙСКИЙ ГОСУДАРСТВЕННЫЙ УНИВЕРСИТЕТ}
			\vspace{0,25 см}
			
 Инженерно-физический факультет
			
 Кафедра автоматизированных систем обработки информации и управления
			\vfill

			\vfill
			
			\textsc{Отчет по практике}\\[5mm]
			
 {\LARGE Вариант 6. \textit{Нахождение ранга матрицы }}
			\bigskip
			
 2 курс, группа 2ИВТ
		\конец{центр}
		\vfill
		
		\newlength{\МЛ}
		\settowidth{\ML}{«\underline{\hspace{0.7 cm}}» \underline{\hspace{2cm}}}
		\hfill\begin{minipage}{0.5\textwidth}
 Выполнил:\\
			\underline{\hspace{\ML}} С.\,А. Гоголев\\
 «\underline{\hspace{0.7 cm}}» \underline{\hspace{2cm}} 2020 г.
		\end{minipage}%
		\bigskip
		
		\hfill\begin{minipage}{0.5\textwidth}
 Руководитель:\\
			\underline{\hspace{\ML}} С.\,В.~Теплоухов\\
 «\underline{\hspace{0.7 cm}}» \underline{\hspace{2cm}} 2020 г.
		\end{minipage}%
		\vfill
		
		\begin{центр}
 Майкоп, 2020 г.
		\конец{центр}
	\end{titlepage}
	\tablefcontents{}
\clearpage
	\section{Введение}
\label{sec:intro}
Найти ранг матрицы
\newlineРангом матрицы называется максимальное число линейно независимых строк, рассматриваемых как векторы.
Отыскание ранга матрицы способом элементарных преобразований (методом Гаусса). Под элементарными преобразованиями матрицы понимаются следующие операции: 1) умножение на число, отличное от нуля; 2) прибавление к элементам какой-либо строки или какого-либо столбца; 3) перемена местами двух строк или столбцов матрицы; 4) удаление "нулевых" строк, то есть таких, все элементы которых равны нулю; 5) удаление всех пропорциональных строк, кроме одной.
Для любой матрицы A всегда можно прийти к такой матрице B, вычисление ранга которой не представляет затруднений. Для этого следует добиться, чтобы матрица B была трапециевидной. Тогда ранг полученной матрицы будет равен числу строк в ней кроме строк, полностью состоящих из нулей.
Ступенчатую матрицу называют трапециевидной или трапецеидальной, если для ведущих элементов a1k1, a2k2, ..., arkr выполнены условия k1=1, k2=2,..., kr=r, т.е. ведущими являются диагональные элементы.\\




\section{Ход работы}
\label{sec:exp:code}
\subsection{Код приложения} 
\label{sec:exp:code}
\begin{дословно}
#include<iostream>
#включить <stdio.h> 
#include <math.h> 
#include <stdlib.h> // Описания функций malloc 
// Прототип функции приведения матрицы к ступенчатому виду. Функция возвращает ранг матрицы
инт гаусс(
 int m, // Число строк матрицы
 int n, // Число столбцов матрицы
 double* a, // Адрес массива элементов матрицы
 double eps // Точность вычислений
);
int main() {
 setlocale(LC_ALL, "rus");
 int m, n, i, j, rang;
 двойной* а;
 двойной eps, det;
 printf("Введите размеры матрицы m, n: ");
 scanf_s("%d%d", &m, &n);
 // выделение памяти под элементы матрицы
 a = (double*)malloc(m * n * sizeof(double));
 printf("Введите элементы матрицы:\n");
 для (i = 0; i) {
 для (j = 0; j) {
 // Вводим элемент с индексами i, j
 scanf_s("%lf", &(a[i * n + j]));
        }
    }
 printf("Введите точность вычислений eps: ");
 scanf_s("%lf", &eps);
 // Вызываем метод Гаусса
 rang = gauss(m, n, a, eps);
 // Ступенчатый вид матрицы
 printf("Ступенчатый вид матрицы:\n");
 для (i = 0; i) {
 // Печатаем i-ю строку матрицы
 для (j = 0; j) {
 printf("%10.3 lf ", a[i * n + j]);
 //Формат %10.3lf означает 10 позиций на печать числа, 3 знака после точки
        }
 printf("\n"); // Перевести строку
    }
 printf("Ранг матрицы = %d\n", rang);
 возврат 0; 
}
/* Приведение матрицы к ступенчатому виду методом Гаусса с выбором максимального элемента в столбце.
 Функция возвращает ранг матрицы*/
инт гаусс(
 int m, // Число строк матрицы
 int n, // Число столбцов матрицы
 double* a, // Адрес массива элементов матрицы
 double eps // Точность вычислений
) {
 int i, j, k, l;
 двойной r;
 i = 0; j = 0;
 while (i < m && j) {
 /* минор матрицы в столбцах 0..j-1 уже приведен к ступенчатому виду, и строк с индексом i-1 содержит ненулевой эл-т
 в столбце с номером, меньшим чем j, Ищем максимальный элемент в j-м столбце, начиная с i-й */
 r = 0,0;
 для (k = i; k) {
 if (abs(a[k * n + j]) > r) {
 l = k; // Запомним номер строки
 r = abs(a[k * n + j]); // и макс. эл-т
            }
        }
 если (r) {
 /* Все элементы j-го столбца по абсолютной
 величине не превосходят eps.
 Обнулим столбец, начиная с i-й строки*/
 для (k = i; k) {
 a[k * n + j] = 0.0;
            }
 ++j; // Увеличим индекс столбца
 continue; // Переходим к следующей итерации
        }
 если (l != i) {
 // Меняем местами i-ю и l-ю строки
 для (k = j; k) {
 r = a[i * n + k];
 a[i * n + k] = a[l * n + k];
 a[l * n + k] = (-r); // Меняем знак строки
            }
        }
 /*abs(a[i*n + k]) > eps. Обнуляем j-й столбец, начиная со строки i+1, применяя элем. преобразования второго рода*/
 для (k = i + 1; k) {
 r = (-a[k * n + j] / a[i * n + j]);
 // К k-й строке прибавляем i-ю, умноженную на r
 a[k * n + j] = 0.0;
 для (l = j + 1; l) {
 a[k * n + l] += r * a[i * n + l];
            }
        }
 ++i; ++j; // Переходим к следующему минору
    }
 return i; // Возвращаем число ненулевых строк
}
\end{дословно}
\section{Пример вставки изображений}
 Скриншот работы работы программы. рис 1, рис 2.
\begin{figure}[h]
    \центрирование
    \includegraphics[width=1.0\textwidth]{Скриншот.JPG}
    \caption{\label{figl}}
\end{рисунок}
\begin{figure}[h]
    \центрирование
    \includegraphics[width=0.4\textwidth]{Снимок koga.JPG}
    \caption{\label{figl}}
\end{рисунок}
\section{Пример библиографических ссылок} 
\subsection{Cписок литературы} 
\begin{перечислить}
    \предмет http://www.cleverstudents.ru/matrix/rank.html
    \пункт http://www.mathprofi.ru/rang_matricy.html
\end{itemize}
\end{документ}
